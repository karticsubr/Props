\documentclass[a4paper,10pt,twocolumn]{article}

% \usepackage{hyperref}

\usepackage{wrapfig}

\usepackage[normalem]{ulem}
\usepackage{multicol}
% \usepackage[style=mla,babel=hyphen,backend=biber]{biblatex}
\usepackage{enumitem}
\usepackage{graphicx}
\usepackage{footmisc}
\usepackage{lastpage}
\usepackage{fancyhdr}
\usepackage{helvet}
\usepackage{microtype}
\usepackage{amssymb,amsfonts}
\usepackage[super]{natbib}
\setlength{\bibsep}{.5ex}
\renewcommand{\familydefault}{\sfdefault}

%%%%%%%%%%%%%
% \oddsidemargin -.20in
% \evensidemargin .80in
% \textwidth 6.8in
% \headheight -.9in
% \topmargin 0.1in
% \textheight 10.2 in
% \footskip -.2in
%%%%%%%%%%%% 


\long\def\symbolfootnote[#1]#2{\begingroup%
\def\thefootnote{\fnsymbol{footnote}}\footnote[#1]{#2}\endgroup} 

\bibliographystyle{unsrt}
% \pagestyle{empty}
% \fancypagestyle{plain}
% \pdfpagewidth 6.8in
% \pdfpageheight 10.2in
\setlength\voffset{-.7 in}
\setlength\hoffset{-.1 in}
\setlength\topmargin{-.1 in}
\setlength\headheight{-.1in}
\setlength\headsep{.4in}
\setlength\textheight{10.2in}
\setlength\textwidth{6.6in}
\setlength\oddsidemargin{0in}
\setlength\evensidemargin{.1in}
\setlength\parindent{0.25in}
% \footskip -.05in
\pagestyle{fancy}
\fancyhf{}
\renewcommand{\headrulewidth}{0pt}
% \cfoot{\thepage\pageref{LastPage}}
\chead{\textit{\small  Page \thepage\ of \pageref{LastPage}}}
\lhead{\textit{\small CAT-SPATS}}
% \chead{Kartic Subr}

\newcommand{\hdg}[1] {\noindent \textbf{#1} }
\newcommand{\Real}  {\ensuremath{\mathbb{R}} }

\rhead{\textit{Kartic Subr}}
% \lhead{\textit{Online Adaptive Sampling of Ray-space  ...}
% for Acquiring Tri-stimulus Plenoptic Functions}
\setlength{\parskip}{.5em}

\begin{document}

{ 
\Large 
\twocolumn[
\begin{center}
% \textbf{Online Adaptive Sampling of Ray-space for} 
% \hfill {\small 
% \textit{Recording all rays of light, }} \\
% \textbf{Acquiring Tri-stimulus Plenoptic 
% Functions} \hfill {\small \textit{ not just pictures from 
% many cameras.}} \\
\textbf{ CAT-SPATS: Crossmodal Analysis and Tracking  \\
of Spatio-temporal Statistics of Targeted Species}
% \textbf{ Interactive-View Imaging of Dynamic Events (IVIDE)\\
%   using a Sparse and Optimised Set of Computational Cameras}
\end{center}
]
}


\begin{figure*}[htbp!]
\begin{center}
 \includegraphics[width=.98\linewidth]{teaser2}
\end{center}
\vspace{-1.5em}
\caption{\emph{The graph (left) indicates that people who live in areas with human-wildlife conflict consider it as an important problem~\cite{owen2013conservation}. Cats such as leopards cause much tension even in urban areas, and are difficult to tranquilise or capture (middle, from live CCTV footage). We propose a pilot project towards predicting the populations of animals from data that is already gathered by conservation agencies. (best viewed on screen)}}
\label{fig:teaser}
\vspace{-1em}
\end{figure*}

\hdg{Human-wildlife conflict in India} India's burgeoning human population has led to unprecedented rates of urbanisation, leaving a mere 5\% of the land area to be earmarked for conservation~\cite{indiacons}. These protected reserves are the last resort for thriving populations of large and threatened mammals. Unfortunately, these fragmented areas are surrounded by human settlements. Each year, thousands of people and hundreds of wild animals lose their lives due to human-wildlife conflict in India. Hundreds of farmers, each year, suffer devastating loss to their property and crops. Conflicts between humans and wild elephants result in about 400 human fatalaities and the deaths of about 50 wild elephants annually. Similarly, there is tension between humans and big cats such as tigers and leopards (about 300 leopards are killed annually). Attacks from wild cats are prevalent even in cities such as Bangalore (around 8 million people). These tensions destabilise~\cite{barua2013hidden} both urban as well as rural populations. People in areas of high wildlife-human conflict are as concerned about this as they are about availability of water, transport and health facilities~\cite{owen2013conservation}. Tracking the spatial statistics of these animals is challenging~\cite{troubletiger}.

\hdg{The dream}
Imagine the capability to dynamically monitor the spatio-temporal distributions of leopards in a particular wildlife reserve. This would serve as an invaluable tool for conservationists across a range of applications: to alleviate human-wildlife conflict, to target anti-poaching measures, to shape informed policies on urbanisation~\cite{gubbi2005new} towards protection of wild habitats, etc. The data towards building such predictive systems needs to stem from multiple sources~\cite{bliznyuk2014nonlinear} such as satellite imagery for assessing foliage and water-bodies~\cite{urthecast}, sensors distributed across the forest (microphones, cameras, etc.), live data streaming in from expert forest rangers in the field, exploratory robotic sensors scoping the forests, etc. 

\hdg{Objectives and scope of this proposal} 
Due to the timeline of 12 months,  as a first step towards the above dream, in this pilot project we propose to focus on crossmodal regression from two sources of data: digital camera-traps and forest ranger data. We shall analyse the spatio-temporal distributions of two behaviourally different species --- elephants and leopards --- in the forest reserves of Karnataka (my home state) in Southern India. We shall automate the classification of images from camera-traps and then use this in conjunction with the data from rangers' logs to estimate spatio-temporal distributions.

\hdg{Expected outcome} The first tangible output of this proposal would be an automatic classification system, that will classify pictures of animals, from a vast database, according to species. This would also include algorithms that can recognise particular individuals (leopards and elephants) accompanied by uncertainty estimates. The second output of this proposal would be a spatio-temporal model built and validated using the camera-trap image database as well as data from forest rangers. This would be an important step towards analysis of populations of threatened and dangerous species that destabilise rural populations.


\hdg{Impact} The algorithm for recognition of individual leopards and elephants would be significant for conservationists in the field who are faced with the arduous task of sifting large databases. We expect to publish the results at venues focusing on pattern recognition. Even more academic impact is expected in our solution of the underlying problem --- non-linear regression of the non-stationary spatio-temporal function with variable uncertainty in the position (domain) of the measurements as well as in the measured values (range).~e.~g.~camera traps are spatio-temporally precise while rangers contribute more data points.

\hdg{Timeliness}
Recently, a number of creative technological solutions have been adapted towards conservation efforts in Karnataka. Forest rangers were issued mobile phones equipped with specially developed applications such as 'Huli' or 'Hejje' (by Sidvin Core-Tech and KeyFalcon Solutions) to facilitate the reporting of animal sightings while the applications automatically performed spatio-temporal tagging. Digital camera traps donated by companies such as CSS Corp became prevalent (about one per 4 sq.km) in the protected forests in Karnataka. 
% A notably different solution, adopted by Wild Seve~\cite{wildseve}, provides a technological platform, along with a toll-free phone number, for farmers to report wildlife conflict incidents. WildSeve was jointly supported by The Rufford Foundation, National Geographic's Big Cats Initiative and Oracle Giving. However the WildSeve is an inherently reparative model which compensates farmers with financial losses in an attempt to pacify their sentiments against co-habitation with wildlife. 
There is an imminent need for \textit{automated information retrieval, archival, scientifically-sound analysis and exploitation} from the various sources of data. 

% \hdg{A note about the sources of data}
% The main sources of data are radio-collars, camera-traps and forest rangers via smartphone applications. 
% Transmitters (radio-collars) are useful means of tracking individuals within a species when the population is small or when the species exhibits clumped (herding) social behaviour.~e.~g.~herds of elephant. However, in regions with thriving populations of animals that pose a threat to human life, such as big cats or lone tuskers, tracking based on a few statistical samples is unacceptable due to dangerously high uncertainty. Further, while tracking using radio-collars is useful for monitoring the individual's health and habits, predicting when they run the risk of human-conflict is non-trivial since it requires active seeking (polling) using a receiver. 
% % 
% Automatically triggered camera-traps can detect individual animals at pre-determined spatial locations where the cameras are installed. One potential problem with camera-traps is determining where to install them: if their installation is too close to human settlement, there is insufficient time to react to a detected threat; if they are installed away from human settlements, the number of false alarms are high. However, their non-invasive sensing capability together with flexibility for coupling with sophisticated software --- such as pattern recognition modules for recognition of species as well as particular individuals (unique striped patterns on tigers) --- make them a valuable form of sensing.
% % 
% A more recent form of tracking has been via the use of smartphones with custom-designed applications such as 'Hejje' (\textit{transl.} pug-mark) for reporting events such as animal sightings, water levels, illegal deforestation, etc. Scores of such phones have been distributed to forest rangers who patrol thousands of kilometres of forest tracks in a month. 

\hdg{Background}
This project will amalgamate the efforts of multiple communities: sensing, modelling, image processing and computer vision~\cite{yu2013automated}, machine learning via computation and conservation. Mr. Ullas Karanth, director of Wildlife Conservation Society India, has pioneered efforts in using modern sensors such as camera traps for the tracking of tigers~\cite{hiby2009tiger} and other species. WCS India has accumulated a massive database with millions of images captured by carefully-placed camera traps over the past two decades. 
% They have collaborated with ecological statisticians to model the behaviour and migration of many animals. 
Unfortunately extracting information by combining models and sampling methods is slow and laborious~\cite{karanth1992population}. More recently researchers in the machine learning community, such as Dr. Sanguinetti, have developed models for analysing spatio-temporal statistics in other contexts~\cite{RIS_0}. Conservationists such as Mr. Sanjay Gubbi are at the forefront of resolving human-wildlife conflict due to regular interaction with affected families. The ideas in this proposal evolved upon discussions with Mr. Karanth, Dr. Sanguinetti and Mr. Gubbi who have all expressed excitement and willingness to collaborate during its execution.  

\hdg{Hypotheses}
We distill the objectives of this proposal down to the following three hypotheses:
\begin{itemize}[topsep=-2.1ex,itemsep=-.7ex,leftmargin=1ex,itemindent=3ex]
 \item [\textbf{H1}.] It is possible to automatically classify leopards and elephants in data from camera traps. In addition, it is possible to use distinctive markings to identify individuals in these speices, along with estimates of uncertainty in the recognition.
 \item [\textbf{H2}.] There exists a duality between uncertainty in position and uncertainty in value of sampled functions. Camera traps are focused on precise location while experienced rangers are more precise in recognising species or individuals. The duality will allow us to integrate both forms of uncertainties within a unified mathematical framework.
 \item [\textbf{H3}.] The underlying regression of the spatio-temporal statistics may be modelled using a Gaussian process by the identification of a suitable, non-stationary covariance function~\cite{paciorek2003nonstationary}.
\end{itemize}
\vspace{2.5ex}
% 
% \hdg{Experimental methods and timeline}
% Simulation: uncertainty.
% Real data: Camera traps and Hejje
% Additional sensors: Microphones with solar-powered FPGA call-detectors.

\hdg{Milestones}
\begin{enumerate} [topsep=-1ex,itemsep=-.1ex,leftmargin=1ex,itemindent=3ex]
 \item [\textbf{M1}] (3  months) Classifier trained, tested and validated using camera-trap data (H1).
 \item [\textbf{M2}] (6  months) Theoretical model of duality (H2).
 \item [\textbf{M3}] (9 months) Develop non-stationary covariance function for the Gaussian process regression model.
 \item [\textbf{M4}] (12 months) Validate results of M1 and M3 using leave-p-out cross validation. Further validation of M3 will be performed by developing a simulator for distributions of fauna in similar fashion to flora~\cite{Bradbury15Guided}, where ground truth distribution data will be available.
 \end{enumerate}
\vspace{2.5ex}

\hdg{Feasibility and risk}
I have garnered the approval of conservationists such as Mr. Sanjay Gubbi from Nature Conservation Foundation India and pioneering scientist Dr. Ullas Karanth (support letter included) who has decades of experience. They have offered to provide access to data as well as collaborative assistance for this project. I am fluent in the local language (Kannada) which will ease interaction with local communities during field trips. The practical risk is that the different data are owned by different agencies. However, the impact will still be considerably invaluable if we are only limited to some sources. The primary risk on the theoretical side is that we are unable to prove the existance of a duality (H2, M2). Fortunately, this is not critical because we may ignore uncertainties in position and proceed with the remainder of the project.

% Combine with data about water levels, urbanisation, foliage, etc. Build simulation models for fauna.


% 
% Motivation: ...
% Human/elephants
% 12-13: 422, 101
% 13-14: 413, 72
% 14-15: 391, 39
% 
% Sources/Engineering/sensors: 
% Rangers/trackers
% Camera traps
% Collars/transmitters
% Microphones
% Tourists
% Multi-sensory UAVs
% 
% Science? Uncertainty modelling, Accumulation / sliding window 
% 
% Long term: Scaling, Satellite imagery for correlating with water sources, urbanisation, etc.
% 
% Collaboration: WCS and NCF
% 
% Timeliness: first Indian leopard census, WildSeve, reports on conflict, huli/hejje, camera-traps
% 
% Hypothesis:
% 
% Experimental methods/timeline:



% \cfoot{\thepage\ of \pageref{LastPage}}
\vspace{-2em}
% \bibliographystyle{ieeetr}
% \begin{multicols}{2}
% {\footnotesize \bibliography{URF14}}
{  \footnotesize \bibliography{RSCG16}}
% {\bibliography{URF14}}
%  \end{multicols}

% \newpage
% asdfs
\end{document}
