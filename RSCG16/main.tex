\documentclass[a4paper,10pt,twocolumn]{article}
\usepackage{hyperref}

\usepackage{wrapfig}

\usepackage[normalem]{ulem}
\usepackage{multicol}
% \usepackage[style=mla,babel=hyphen,backend=biber]{biblatex}
\usepackage{enumitem}
\usepackage{graphicx}
\usepackage{footmisc}
\usepackage{lastpage}
\usepackage{fancyhdr}
\usepackage{helvet}
\usepackage{amssymb,amsfonts}
\usepackage[super]{natbib}
\setlength{\bibsep}{.5ex}
\renewcommand{\familydefault}{\sfdefault}

%%%%%%%%%%%%%
% \oddsidemargin -.20in
% \evensidemargin .80in
% \textwidth 6.8in
% \headheight -.9in
% \topmargin 0.1in
% \textheight 10.2 in
% \footskip -.2in
%%%%%%%%%%%% 


\long\def\symbolfootnote[#1]#2{\begingroup%
\def\thefootnote{\fnsymbol{footnote}}\footnote[#1]{#2}\endgroup} 

\bibliographystyle{unsrt}
% \pagestyle{empty}
% \fancypagestyle{plain}
% \pdfpagewidth 6.8in
% \pdfpageheight 10.2in
\setlength\voffset{-.7 in}
\setlength\hoffset{-.1 in}
\setlength\topmargin{-.1 in}
\setlength\headheight{.1in}
\setlength\headsep{.1in}
\setlength\textheight{10.6in}
\setlength\textwidth{6.9in}
\setlength\oddsidemargin{-.2in}
\setlength\evensidemargin{.2in}
\setlength\parindent{0.25in}
% \footskip -.05in
\pagestyle{fancy}
\fancyhf{}
\renewcommand{\headrulewidth}{0pt}
% \cfoot{\thepage\pageref{LastPage}}
\chead{\textit{\small  Page \thepage\ of \pageref{LastPage}}}
\lhead{\textit{\small CAT}}
% \chead{Kartic Subr}

\newcommand{\hdg}[1] {\noindent \textbf{#1} }
\newcommand{\Real}  {\ensuremath{\mathbb{R}} }

\rhead{\textit{Kartic Subr}}
% \lhead{\textit{Online Adaptive Sampling of Ray-space  ...}
% for Acquiring Tri-stimulus Plenoptic Functions}
\setlength{\parskip}{.5em}

\begin{document}

{ 
\Large 
\twocolumn[
\begin{center}
% \textbf{Online Adaptive Sampling of Ray-space for} 
% \hfill {\small 
% \textit{Recording all rays of light, }} \\
% \textbf{Acquiring Tri-stimulus Plenoptic 
% Functions} \hfill {\small \textit{ not just pictures from 
% many cameras.}} \\
\textbf{ Crossmodal Animal Tracking (CAT): \\
Predicting Spatio-temporal Statistics of Dynamic Species}
% \textbf{ Interactive-View Imaging of Dynamic Events (IVIDE)\\
%   using a Sparse and Optimised Set of Computational Cameras}
\end{center}
]
% \vspace{-4em}
}

% \begin{figure*}
%  \includegraphics[width=\linewidth]{URF14HLD2}
% %  \vspace{-2em}
% \caption{\emph{Visual summary of A) novelty, B) application context and C) and D) key insight to sampling problem.}}
% \label{fig:Plan}
%  \vspace{-1.5em}
% \end{figure*}

\hdg{Natural habitats and urbanisation}

\hdg{Monitoring population distributions}
Imagine a system that enables real-time hypothesisation of the spatial population density of any queried species of animals. Such a capability would serve as an invaluable tool for conservationists. The data would be useful for decisions to mitigate human-wildlife conflict, to target anti-poaching measures, to drive policies on urbanisation and protection of wild habitats, etc. I want to drive a pilot project towards the development of methodologies that are required for the development of such a system. In this pilot projcet I shall focus on two behaviourally different species --- elephants and leopards --- in the reserved forests of Karnataka (my home state) in Southern India. 


\hdg{Timeliness}
The stupendous rate of development of information technology in Bangalore, the capital of Karnataka, has fostered the development of a number of creative technologies towards application in conservation. First, forest rangers were issued mobile phones with specially developed applications such as 'Huli' or 'Hejje' (by Sidvin Core-Tech) installed on them to report animal sightings combined with automatic tagging of the time along with coordinates. Next, digital camera traps donated by companies such as CSS Corp (\url{https://www.csscorp.com/}) became prevalent (about one per 4 sq.km) in the protected forests in Karnataka. While many of these developments provide useful measuring devices, there is an imminent need for a scientifically sound archiving, analysis and exploitation of the data from such sources. A notable exception to this trend is WildSeve~\cite{}, a recent project that provides a technological platform, along with a toll-free phone number, for farmers to report wildlife conflict incidents. WildSeve was jointly supported by The Rufford Foundation, National Geographic's Big Cats Initiative and Oracle Giving.

\hdg{Data from multiple sources}

\hdg{The scientific problem}

\hdg{Hypothesis}

\hdg{Experimental methods and timeline}

\hdg{Foundation for future projects}

\hdg{Suitability for endeavour}

% 
% Motivation: ...
% Human/elephants
% 12-13: 422, 101
% 13-14: 413, 72
% 14-15: 391, 39
% 
% Sources/Engineering/sensors: 
% Rangers/trackers
% Camera traps
% Collars/transmitters
% Microphones
% Tourists
% Multi-sensory UAVs
% 
% Science? Uncertainty modelling, Accumulation / sliding window 
% 
% Long term: Scaling, Satellite imagery for correlating with water sources, urbanisation, etc.
% 
% Collaboration: WCS and NCF
% 
% Timeliness: first Indian leopard census, WildSeve, reports on conflict, huli/hejje, camera-traps
% 
% Hypothesis:
% 
% Experimental methods/timeline:



% \cfoot{\thepage\ of \pageref{LastPage}}
\vspace{-1em}
% \bibliographystyle{ieeetr}
% \begin{multicols}{2}
% {\footnotesize \bibliography{URF14}}
{ \footnotesize \bibliography{URF14}}
% {\bibliography{URF14}}
%  \end{multicols}

% \newpage
% asdfs
\end{document}
